\documentclass[a4,12pt]{article}
\begin{document}
\author{Dilip Paneru}
\date{}
\title{Grover's search}
\maketitle
Although Grover's search algorithm doesn't offer the spectacular exponential speed-up of Shor's factoring algorithm,it does offer a polynomial speed-up over classical search algorithms.For searching among N items it takes $O(\sqrt{N})$ steps to find the solution.The search algorithm is formulated in the form of finding x from the list of N items such that $\ f(x)=1$. The steps of Grover's search algorithm can be explained as follows:
\begin{enumerate}
\item First of all we start by preparing our qubits in the initial superposition of $\ \frac{1}{\sqrt{N}} \sum_{x=0}^{N-1}|x\rangle$   by applying Hadamard transform to the state  $\ |0\rangle $
\item Then we apply the Quantum oracle such that we obtain the state $\ \frac{1}{\sqrt{N}} \sum_{x=0}^{N-1}(-1)^{f(x)}|x\rangle$.This can be done by applying the controlled not gate to $|-\rangle$ state controlled on the value of $\ f(x)$, such that $(-1)^{f(x)}|x\rangle|-\rangle$ is obtained from the state $|x\rangle|-\rangle$. This process is also called phase inversion because the phase of the solution states are inverted.
\item Next step is amplifying the probability amplitudes of the solution states. This can be done by reflecting the amplitudes about their mean such that $ \alpha_x$ gets transformed into $ 2\frac{\sum_{x}\alpha_x}{N} - \alpha_x $ . This corresponds to act of reflecting the states about the state
$ |u\rangle = \frac{1}{\sqrt{N}} \sum_{x=0}^{N-1}|x\rangle$. This can be seen as follows:
\\ After reflection, the state obtained is $\langle\psi|u\rangle-(|\psi\rangle - \langle\psi|u\rangle|u\rangle)$ as the component parallel to $|u\rangle$ is kept same and the components orthogonal to it are inverted in phase.After simplifying we obtain $   \sum_{x=0}^{N-1}(2\frac{\sum_{x}\alpha_x}{N}-\alpha_x)|x\rangle$ 
The overall operation is $2|u\rangle \langle u|-I$
This process is carried out in three further steps:
\begin{enumerate}
	\item Transforming $|u\rangle$ into $|0\rangle$ by applying $H^{\otimes n}$
	\item Reflecting $|\psi\rangle$ about $|0\rangle$ by applying $2|0\rangle \langle 0|-I$
	\item Transforming $|0\rangle$ back into $|u\rangle$by applying $H^{\otimes n}$	
\end{enumerate}
\item Geometrically,in the space spanned by the normalized superposition of solution states $|v\rangle$ and the normalized superposition of non-solution states $|w\rangle$,the initial state can be written as $ |\psi\rangle = \cos({\frac{\theta}{2}})|w\rangle + \sin({\frac{\theta}{2}})|v\rangle$,where there are M solutions then 
$\cos({\frac{\theta}{2}}) = \sqrt{\frac{M-N}{N}}$ and $\sin({\frac{\theta}{2}}) = \sqrt{\frac{M}{N}}$. The phase inversion is reflects the state about $|w\rangle$ and the operation $2|u\rangle \langle u|-I$ is reflects the state about $|u\rangle$. The net result is the rotation of the state towards $|w\rangle$ by angle $\theta$. 
\item The iterations need to be performed $O(\sqrt{\frac{N}{M}})$ times to obtain a solution state under measurement with high probability.
\end{enumerate}

\begin{thebibliography}{9}
	
	\bibitem{} 
	Umesh vazirani: Grover's Search Algorithm.
	\\\texttt{https://d37djvu3ytnwxt.cloudfront.net/assets/courseware/v1/faff66d79487fc4deee6246aa50cffef/c4x/BerkeleyX/CS-191x/asset/chap7.pdf}
	
	\bibitem{} 
	Michael A. Nielsen and Isaac L. Chuang. 
	\textit{Quantum Computation and Quantum Information}. 
	Cambridge University Press,New York,2010.
	
	\bibitem{} 
	Philip Kaye, Raymond Laflamme and Michele Mosca. 
	\textit{An Introduction to Quantum Computing}. 
	Oxford University Press,New York,2007.
	
\end{thebibliography}


\end{document}